\documentclass[12pt]{article}
\usepackage{graphicx}
\usepackage{ulem}
\usepackage{fancyvrb}
\usepackage{url}
\usepackage{thm}
\title{CS442 Section 001 Question 5}

\begin{document}
\maketitle

\section{Explaining the ``Why?"}

Theorem 3.5.4 does not hold because the evaluation rules $[S-L]$ and $[S-R]$ are ambiguous.
Namely, we can choose either one to evaluate $t_1 \oplus t_2$, because both rules only expect 2 terms ($t$).

\section{Proof for the required statement}

First, we begin with a lemma.
\begin{lemma}
\label{lemma:diamond}
Diamond property.
Namely, if $r \rightarrow s$ and $r \rightarrow t$, with $s \neq t$, then there is some term $u$ such that $s \rightarrow u$ and $t \rightarrow u$.
\end{lemma}
\begin{proof}
Let's do induction on the derivation of $t$.
\begin{itemize}
\item
Case $[S-P]$:
Not applicable because $n_1 + n_2$ is a value.
\item
Case $[S-L]$:
Since $t$ must have the form $t_1 \oplus t_2$,
we know that the 2 derivations, $s$ and $t$, must have the form
\begin{align}
t'_1 \oplus t_2 \text{ and } t''_1 \oplus t_2
\end{align}
, respectively, where $t \rightarrow t'_1$ and $t \rightarrow t''_1$.
Since $t'_1$ and $t''_1$ are sub-terms of $t$, the inductive hypothesis applies.
Namely, for some $t'''_1$,
\begin{align}
t'_1 \rightarrow t'''_1 \text{ and } t''_1 \rightarrow t'''_1
\end{align}
Thus, from either $s$ or $t$, we can apply $[S-L]$ again to get $u = t'''_1 \oplus t_2$.
Hence, $s \rightarrow u$ and $t \rightarrow u$.
\item
Case $[S-R]$:
Symmetric argument to $[S-L]$, for $t_2$ instead.
\end{itemize}
As required.
\end{proof}
Now, we prove the main statement.
\begin{claim}
If $t \rightarrow^* n_1$ and $t \rightarrow^* n_2$, then $n_1 = n_2$.
\end{claim}
\begin{proof}
By induction of the length of derivation:
\begin{itemize}
\item
Case length 0 ($value$):
Automatically holds because $n \rightarrow^* n$ by definition of $\rightarrow^*$.
\item
Case length 1 ($[S-P]$):
Also automatically hold due to having a small step in this case; hence, lemma \ref{lemma:diamond} follows.
\item
Case length $> 1$ ($[S-L]$ or $[S-R]$):
We have $t \rightarrow^* n_1$ and $t \rightarrow^* n_2$.
Consider the next derivation of $t$, say $t \rightarrow t'_1$ and $t \rightarrow t''_1$, assuming $t'_1 \neq t''_1$.
Lemma \ref{lemma:diamond} can be applied, which we have $t \rightarrow^* u_1$.
Thus, we have a ``checkpoint" (I think of this like a database checkpoint), and we can discard $t$ and continue the evaluation from $u_1$.
Then, we can repeat the process until we get a number, i.e. in normal form.
In particular,
\begin{align}
& t \rightarrow^* u_1 \rightarrow^* \cdots \rightarrow^* u_{k-1} \rightarrow s_1 \rightarrow n_1 \\
\text{and } & t \rightarrow^* u_1 \rightarrow^* \cdots \rightarrow^* u_{k-1} \rightarrow s_2 \rightarrow n_2
\end{align}
Finally, we apply lemma \ref{lemma:diamond} on $u_{k-1} \rightarrow s_1$ and on $u_{k-1} \rightarrow s_2$ in order to deduce that $n_1 = n_2$.
\end{itemize}
\end{proof}
\end{document}
